%% This is annot.tex.
%% 
%% You'll need to change the title and author fields to reflect your
%% information.
%%
%% Author: Titus Barik (titus@barik.net)
%% Homepage: http://www.barik.net/sw/ieee/
%% Reference: http://www.ctan.org/tex-archive/info/simplified-latex/

\documentclass [11pt]{article}
\usepackage{url}
\usepackage{longtable}
\usepackage{amsmath}
\usepackage{amssymb}
%\usepackage{mathalpha}
%\usepackage{upgreek}

%\usepackage[
%%backend=biber, 
%natbib=true,
%style=numeric,
%sorting=none
%]{biblatex}

%\addbibresource{biblatex-examples.bib}

\title{Moral Reasoning AI Automobiles\\\medskip An Annotated Bibliography}
\author{Th\'er\`ese Smith\\ (tmsmith@wpi.edu)\\Worcester Polytechnic Institute}

\begin{document}
\maketitle

Automobiles controlled in the domain of speed and direction by artificial intelligence pose moral issues around response to accidents developing.
We want to develop terms of discourse about moral reasoning by artificial intelligence such that this discourse can include participants throughout society. For example, persons who purchase a motor vehicle, and are today responsible for driving it, and participate in how accident situations are handled, should know at the same level of detail as they do today for their own conduct, about the anticipated behavior of their automobile.

This project used  \ldots
.

For this literature review, 
\begin{itemize}
\item The \textbf{focus} of the inclusive involvement in moral reasoning is to develop an ontology of terms of moral discourse about artificial intelligence controlling automobiles in accident resolution situations to which the general public has been facilitated in contributing.

\item The \textbf{goal} of the literature search is to find out to what extent this has already been done, and if it is not finished, then to develop an understanding of how we may extend the work.
.
\item The (intended)
contribution is:
a process of establishing a time sequence of (versioned) components, suitable for oversight by a committee analogous to those in the ethical oversight in the National Human Genome Research Institute, described at \url{https://www.genome.gov/10001618/the-elsi-research-program/}. The components include training data, test data and an ontology of terms of discourse.


\item What was \textbf{achieved so far} is to find 
\begin{itemize}
	\item  
\end{itemize}
\item More progress could be made in the direction by following the reference trail.
\item The \textbf{coverage style} used is purposeful sampling, beginning with keyword search on \url{scholar.google.com}.
\end{itemize}

%\begin{tabular}{|p{2cm}|p{10cm}|}\hline
\begin{center}
\begin{longtable}{|p{3cm}|p{10cm}|}
\caption[Definitions]{Definitions} \label{grid_mlmmh1} \\

\hline \multicolumn{1}{|c|}{\textbf{Term}} & 
 
\multicolumn{1}{c|}{\textbf{Definition}} \\ \hline 
\endfirsthead

\multicolumn{2}{c}%
{{\bfseries \tablename\ \thetable{} -- continued from previous page}} \\
\hline \multicolumn{1}{|c|}{\textbf{Term}} &
 
\multicolumn{1}{c|}{\textbf{Definition}} \\ \hline 
\endhead

\hline \multicolumn{2}{|r|}{{Continued on next page}} \\ \hline
\endfoot

\hline \hline
\endlastfoot
Term & Definition \\\hline\hline

 &  \\\hline


 & l\\\hline


\end{longtable}
\end{center}
%\end{tabular}



\begin{center}
\begin{longtable}{|p{2cm}|p{10cm}|}
\caption[Definitions]{Definitions} \label{grid_mlmmh} \\

\hline \multicolumn{1}{|c|}{\textbf{Symbol/Mathematization}} & 
 
\multicolumn{1}{c|}{\textbf{Definition}} \\ \hline 
\endfirsthead

\multicolumn{2}{c}%
{{\bfseries \tablename\ \thetable{} -- continued from previous page}} \\
\hline \multicolumn{1}{|c|}{\textbf{Symbol/Mathematization}} &
 
\multicolumn{1}{c|}{\textbf{Definition}} \\ \hline 
\endhead

\hline \multicolumn{2}{|r|}{{Continued on next page}} \\ \hline
\endfoot

\hline \hline
\endlastfoot
Symbol/Mathematization & Definition \\\hline\hline
$ $   &  \\\hline 


\end{longtable}
\end{center}
%\end{tabular}
  
\begin{itemize}
\item  
\end{itemize}

\section{Questions}
\begin{enumerate}
	\item  Presuming that people can agree on preferable outcomes some specific choices, such as harmless vs. significant harm, to what extent can we find agreement on preferable outcomes of specific choices?

 

 

\end{enumerate}
\nocite{*}
\bibliographystyle{IEEEannot}
\bibliography{IQPbibliography}

\end{document}
